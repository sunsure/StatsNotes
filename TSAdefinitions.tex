Time Series Analysis  - Definitions

Time Series Analysis  - Definitions
Autocorrelation function
Autoregressive model
Backwards shift operator and difference operator
Lag Operator
Ljung–Box test statistic
Moving Average Models

Autocorrelation function
The autocovariance function is measured in squared units, so that the values obtained depend on the absolute size of the measurements. We can make this quantity
independent of the absolute sizes of xn by defining a dimensionless quantity, the autocorrelation function.

The autocorrelation function (ACF) of a stationary process is defined by: k=corr(xt,xt+k)=k0
The ACF of a purely indeterministic process satisfies k as k .
Autoregressive model
The notation AR(p) refers to the autoregressive model of order p. The AR(p) model is defined as 
Backwards shift operator and difference operator

The backwards shift operator, B, acts on the process X to give a process BX such that: (BX)t=Xt-1

If we apply the backwards shift operator to a constant, then it doesn’t change it: B=

The difference operator, , is defined as = 1- B , or in other words: (X)t=Xt-Xt-1

Both operators can be applied repeatedly. For example:

	(B2X)t=(B(BX))t=(BX)t-1=Xt-2

	(2X)t= (X)t- (X)t-1=Xt-2Xt-1+Xt-2

	

 Lag Operator
In time series analysis, the lag operator or backshift operator operates on an element of a time series to produce the previous element. 

For example, given some time series X={X1,X2,}then LXt=Xt-1  for all   where L is the lag operator
 
Ljung–Box test statistic
 
Q=T(T+2)k=1srk2/(T-k)
 
T = number of observations
s = number of coefficients to test autocorrelation
rk = autocorrelation coefficient (for lag k)
Q = portmanteau test statistic.
 
Moving Average Models
The moving average (MA) model is common approach for modeling univariate time series models. Xt=+t+1t-1++qt-q


