MS4217 Queuing Theory


MS4217 Queuing Theory
Notation
Simple Queues (M/M/1)
Steady State Equations
Queueing Time
Length of Queue
Multiple Servers (M/M/n)
Exam Question

%--------------------------------------------------------------------------------------------------------%

In general, a queue is characterised by

(a) the arrival process
(b) the service time distribution
(c) the number of servers
(d) the queue discipline.


Notation
The simple queue is designated M|M|1, which means there are Markov arrivals, Markov service and 1 server. 

M|M|2 implies Markov arrivals, Markov service and 2 servers. M|M|n implies Markov arrivals, Markov service and n servers.


Simple Queues (M/M/1)
Assumptions for the M|M|1

(a) Poisson arrivals, rate , i.e. M()
(b) Exponential Service rate , i.e. M()
(c) Assumptions (a) and (b) are independent
(d) 1 server.

The probability that the queue contains x customers at time t is denoted px(t).

Steady State Equations
p1+p0= 0 for x = 0

pn=pn-1=np0

 is known as the traffic intensity

p0= 1-

		
Axiom of probability; Sum of probabilities equals one.
x=0px= 1  
	                       
		
x=0xp0=p0x=0x= 1
Geometric series: x=0x=11-

p0= 1-

px=x( 1-)
Queueing Time  
We make the assumption that the queue is in the steady state.

	Queuing Time=Waiting Time + Service Time.

Length of Queue
E(N) =n=1npn=n=1n(1-)n

		n=1n(1-)n =n=1nn-n=1nn+1
		
	      
		 =n=1n=1-

%-------------------------------------------------------------------------------------------------------------% 


Multiple Servers (M/M/n)
Two scenarios
1 ) There are free servers. No need to queue.
2)  There are no free servers. There is a need to queue.



Exam Question 
In a M/M/1 queue, arrivals form a Poisson process with rate  and service times are i.i.d. random variables following the Negative Exponential distribution with parameter: f(t) =exp(-t) where in the queuing process ( < ).


(a) Write down the steady-state difference equations for the queue size, X, and solve them for Pr(X=x) the probability distribution of the queue size (x = 0,1,2).

(b) A new customer arrives. Derive the queuing time distribution given that there are x(0) customers ahead in the queue.

(c) Hence derive the unconditional queueing time distribution. Identify all of the various distributions derived.




