Stats Education: Structural Equation Modelling (SEM)


Terminology
Steps of SEM
Path Diagrams

Structural Equation Modeling is a very general, very powerful multivariate analysis technique that includes specialized versions of a number of other analysis methods as special cases. 

SEM is an umbrella of 3 processes:

1. Path Analysis: Analysis of structural models of observed variables
2. Confirmatory Factor Analysis:  Analysis of a priori measurement models where both the number of factors and their
correspondence to the indicators are explicitly specified
3. Structural Regression Models: The synthesis of path and measurement models

Terminology
Latent variable: A variable in the model that is not measured
Exogenous variable: Variable that is not caused by another variable in the model, but usually causes one or more variables in the model.
X is the number of exogeneous variables.
Endogenous variable: Variable that is caused by one or more variables in the model, but an endogenous variable may cause another endogenous variable. N is the number of endogeneous variables.
Disturbance: Unspecified causes of endogenous variables, akin to an error or residual
Structural Model: Model constructed
 
Simultaneous equation bias refers to the overestimation or underestimation of the structural parameters obtained from the OLD to the structure equations of an SEM model. This bias results because those endogeneous variables of the system which are also explanatory variables are correlated with the error terms, thus violating the assumptions of the OLS.
Identification
Identification refers to the possibility of calculating the structural parameters of a simultaneous-equations model from the reduced-form parameters.
An equation of a system can be identified, overidentified or underidentified. The system as a whole is exactly identified if all of its component equations are exactly identified.
Steps of SEM
1) Specification
2) Identification
3) Data collection
4) Estimation

Path Diagrams
Path Diagrams play a fundamental role in structural modeling. Path diagrams are like flowcharts. They show variables interconnected with lines that are used to indicate causal flow.

One can think of a path diagram as a device for showing which variables cause changes in other variables. However, path diagrams need not be thought of strictly in this way. They may also be given a narrower, more specific interpretation.

Estimation
Indirect Least Squares (ILS) is a method of calculating structural parameter values for exactly identified equations.
Two stage least squares (2SLS) is a method of estimating consistent structural parameters for over-identified equations. For exactly identified equations, 2SLS gives the same result as ILS, but gives the the standard errors of the estimated structural parameters.

