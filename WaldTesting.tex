


Wald Test

Wald Test
The Wald test is a way of testing the signi®cance of particular explanatory variables in a statistical model. In logistic regression we have a binary outcome variable and one or more explanatory variables. For each explanatory variable in the model there will be an associated parameter.
The Wald test, described by Polit (1996) and Agresti (1990), is one of a number of ways of testing whether the parameters associated with a group of
explanatory variables are zero.

If for a particular explanatory variable, or group of explanatory variables, the Wald test is significant, then we would conclude that the parameters associated with these variables are not zero, so that the variables should be included in the model. If the Wald test is not significant then these explanatory variables can be omitted from the model. When considering a single explanatory variable, Altman (1991) uses a t-test to check whether the parameter is significant. 

For a single parameter the Wald statistic is just the square of the t-statistic and so will give exactly equivalent results.
An alternative and widely used approach to testing the significance of a number of explanatory variables is to use the likelihood ratio test. This is
appropriate for a variety of types of statistical models. Agresti (1990) argues that the likelihood ratio test is better, particularly if the sample size is small or the parameters are large.


