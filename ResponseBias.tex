Response bias
Response bias is a type of cognitive bias which can affect the results of a statistical survey if respondents answer questions in the way they think the questioner wants them to answer rather than according to their true beliefs. This may occur if the questioner is obviously angling for a particular answer ("push polling") or if the respondent wishes to please the questioner by answering what appears to be the "morally right" answer.
 
This occurs most often in the wording of the question. Response bias is present when a question contains a leading opinion. For example, saying "Given that at the age of 18 people are old enough to fight and die for your country, don't you think they should be able to drink alcohol as well?" yields a response bias. It is better to say "Do you think 18-year-olds should be able to drink alcohol?".
 
It also occurs in situations of voluntary response, such as phone-in polls, where the people who care enough to call are not necessarily a statistically representative sample of the actual population.
Non-Response Bias
Non-response bias is not the opposite of "response bias" and is not a type of cognitive bias: it occurs in a statistical survey if those who respond to the survey differ in the outcome variable (for example, evaluation of the need for financial aid) from those who do not respond. Often, the differences, which may include race, gender or socioeconomic status, are reported and/or accounted for through statistical modelling in any publication of the results.

