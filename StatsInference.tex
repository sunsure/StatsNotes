

MA4125 Lecture 7b
Confidence Intervals
Commonly used analyses
MA4125 Lecture 7b
 
Statistical inference describe systems of procedures that can be used to draw conclusions from datasets arising from systems affected by random variation. The main types of procedures are confidence intervals and hypothesis tests. Hypothesis tests can be conducted by using considering the p-value, or comparing the test statistic and the critical value.
 
Despite the ubiquity of p-value tests, this particular test for statistical significance has come under heavy criticism due both to its inherent shortcomings and the potential for misinterpretation.

The p-value is the probability of obtaining a test statistic at least as extreme as the one that was actually observed,
assuming that the null hypothesis is true.
 
The lower the p-value, the less likely the result is if the null hypothesis is true, and consequently the more "significant" the result is, in the sense of statistical significance. One often accepts the alternative hypothesis, (i.e. rejects a null hypothesis) if the p-value is less than 0.05 or 0.01, corresponding respectively to a 5% or 1% chance of rejecting the null hypothesis when it is true (Type I error).
 
A small p-value that indicates statistical significance does not indicate that an alternative hypothesis is automatically correct; there are additional tests which may be performed in order to make a more definitive statement about the validity of the null hypothesis, such as some "goodness of fit" tests.
 
A hypothesis is a claim or statement about a property of definition.
 
A hypothesis test (or test of significance) is a standard procedure for testing a claim about a property of a population.
 
 
Error Types
 
Type I and type II Error
 
Power of a test
 
To descrease both  and 
 
 
Power of a hypothesis Test is the probability of rejecting a false null hypothesis
Confidence Intervals
 
Confidence interval
 
A confidence ellipse is a two dimensional
 
A confidence ellipsoid is a multidimensional
 
P-value
 

Commonly used analyses
 
inference for a single proportion,
inference for a single mean,
inference for the difference between two proportions,
inference for the difference between two means;
 
 
The Paired t-test
The chi-squared test applied to contingency tables
The Anderson Darling Test of Normality
The Kolmogorov Smirnov test for distributions
The Grubb's Test for Outliers
 
 
 
