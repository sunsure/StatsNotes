Base rate fallacy.
The base rate fallacy consists of failing to take into account prior probabilities (base rates) when computing conditional probabilities from other conditional probabilities. It is related to the Prosecutor's Fallacy.
 
For instance, suppose that a test for the presence of some condition has a 1% chance of a false positive result (the test says the condition is present when it is not) and a 1% chance of a false negative result (the test says the condition is absent when the condition is present), so the exam is 99% accurate.
 
What is the chance that an item that tests positive really has the condition?
 
The intuitive answer is 99%, but that is not necessarily true: the correct answer depends on the fraction f of items in the population that have the condition.
 
The chance that an item tests positive is 0.99×f/(0.99×f + 0.01×(1−f)), which could be much smaller than 99% if f is small.
